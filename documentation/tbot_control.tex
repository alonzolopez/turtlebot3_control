\documentclass{article}
\usepackage[utf8]{inputenc}
\usepackage{amsmath}
\usepackage{hyperref}
\def\code#1{\texttt{#1}}
\usepackage{xcolor}

\pagecolor[rgb]{0,0,0} %black

\color[rgb]{0.8,0.8,0.8} %grey


\title{Turtlebot EKF and Control}
\author{Alonzo Lopez}
\date{March 2021}

\begin{document}

\maketitle

\section{System Description}
\subsection{State}
\begin{equation}
    X = \begin{bmatrix}
    x & y & \theta
    \end{bmatrix}^T
\end{equation}

\subsection{Actions}
We can send commands in the form of \href{https://docs.ros.org/en/api/geometry_msgs/html/msg/Twist.html}{ROS Twist} messages containing desired linear and angular velocity. 
\begin{equation}
    X = \begin{bmatrix}
    v & \omega
    \end{bmatrix}^T
\end{equation}
where $v = \dot x$ and $\omega = \dot \theta$.
Note that though we can send other velocities (e.g. $v_y$ and $v_z$ and angular rates about $x$ and $y$), those velocity components are ignored and only the two components,
$v_x$ and angular velocity about $z$, or $\omega$ are extracted to command the forward velocity of the robot and the angular turning velocity, respectively.
To simplify the notation going forward, we use $v$ instead of $v_x$.
The velocity commands are then split across the two wheels of the differential drive wheel base by the 
\href{https://github.com/ros-controls/ros_controllers/wiki/diff_drive_controller}{ROS Differential Drive Controller}.

\subsection{Dynamics}
To describe the dynamic motion of the Turtlebot, we use the \href{http://faculty.salina.k-state.edu/tim/robotics_sg/Control/kinematics/unicycle.html}{simple Unicycle model}
(also see \href{http://planning.cs.uiuc.edu/ch13.pdf}{LaValle's chapter on differential models})
The continuous-time equations of motion are 
\begin{equation}
    \begin{split}
        \dot x & = v \; cos(\theta) \\
        \dot y & = v \; sin(\theta) \\
        \dot \theta & = \omega \\
    \end{split}
\end{equation}
The discretized dynamics are
\begin{equation}
    \begin{split}
        x_{k+1} & = x_k + v_k \; cos(\theta_k) \; dt \\
        y_{k+1} & = y_k + v_k \; sin(\theta_k) \; dt \\
        \theta_{k+1} & = \theta_k + \omega_k \; dt \\
    \end{split}
\end{equation}
% \begin{equation}
%     \begin{split}
%         x_{k+1} & = x_k + v_k \; cos(\theta_k) \; dt \\
%         y_{k+1} & = y_k + v_k \; sin(\theta_k) \; dt \\
%         \theta_{k+1} & = \theta_k + \omega_k \; dt \\
%         \dot x_{k+1} & = v_{k+1} \; cos(\theta_{k+1}) \\
%         \dot y_{k+1} & = v_{k+1} \; sin(\theta_{k+1}) \\
%         \dot \theta_{k+1} & = \omega_{k+1}\\
%     \end{split}
% \end{equation}
It's important to note that we assume our control action for $u = \begin{bmatrix}
    v & \omega
\end{bmatrix}^T$ takes effect immediately as opposed to a more realistic model that incorporates acceleration.

\begin{equation}
    X_{k+1} = f(X_{k}) + g(X_{k}, u_{k})
\end{equation}
\subsection{Sensors}

\section{Experiment}

\subsection{Truth in Simulation}

\end{document}
